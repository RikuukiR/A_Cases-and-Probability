% 第1節 場合の数
\section{場合の数}
\subsection{集合の要素の個数}
% 割愛

\subsection{場合の数}
% 割愛

\subsection{順列}
% 順列
\begin{definitionbox}[def:順列]{\textbf{順列}}
    \label{def:順列}
    いくつかのものを順に1列に並べるとき, その並びの1つ1つを \textbf{順列} という.

    一般に, \uwave{異なる $n$ 個のものから異なる $r$ 個を取り出して}並べる順列を
    \begin{center}
        $\bm{n}$\textbf{個から}$\bm{r}$\textbf{個取る順列}
    \end{center}
    といい, その総数を $_{\bm{n}}\mathbf{P}_{\bm{r}}$ $^{*}$ で表す.
\end{definitionbox}

\begin{theorembox}[thm:順列の総数①]{\textbf{順列の総数① $_{\bm{n}}\mathbf{P}_{\bm{r}}$}}
    \label{thm:順列の総数①}
    $n$ 個から $r$ 個取る順列の総数 $_{n}\mathrm{P}_{r}$ は次の式で表される.

    \begin{align}
        _{\bm{n}}\mathbf{P}_{\bm{r}} = \bm{n \cdot (n-1) \cdot (n-2) \cdots \cdots (n-r+1)}
        \label{eq:順列の総数①}
    \end{align}
\end{theorembox}
\vspace{1em}

順列の総数 ${}_n\mathrm{P}_r$ の式で, 特に $r = n$ のときは等式

\begin{align*}
    _{n}\mathrm{P}_{n} = n(n-1)(n-2) \cdots \cdots 3 \cdot 2 \cdot 1 \tag{*}
\end{align*}

が得られる.

% 階乗
\begin{definitionbox}[def:階乗]{\textbf{階乗}}
    \label{def:階乗}
    $(*)$ の右辺は, $1$ から $n$ までのすべての自然数の積である.

    これを $n$ の \textbf{階乗} といい, $\bm{n!}$ で表す.
\end{definitionbox}


\vfill

\noindent
\rule{\textwidth}{0.4pt}

\noindent
$^{*}$ ${}_n\mathrm{P}_r$ の P は 順列を意味するPermutation の頭文字である.\\

\newpage

\begin{theorembox}[thm:階乗]{\textbf{階乗}}
    \label{thm:階乗}
    Definition\ref{def:階乗} より, $n$ の階乗 $n!$ は次の式で表される.

    \begin{align}
        _{\bm{n}}\mathbf{P}_{\bm{r}} \bm{=} \bm{n!} \bm{=} \bm{n (n-1) (n-2) \cdots \cdots 3 \cdot 2 \cdot 1}
        \label{eq:階乗}
    \end{align}
\end{theorembox}
\vspace{1em}

Definition\ref{def:順列}より, 一般に, 次のことがいえる.

\[\textbf{異なる} \bm{n} \textbf{個のものすべてを並べる順列の総数は}\;\; \bm{n!} \;\;\textbf{である}\]

また, 順列の総数 ${}_n\mathrm{P}_r$ の式で, $r < n$ のときについて考える.
\vspace{1em}

% 証明


\begin{theorembox}[thm:順列の総数②]{\textbf{順列の総数② $_{\bm{n}}\mathbf{P}_{\bm{r}}$}}
    \label{thm:順列の総数②}

    \begin{align}
        _{\bm{n}}\mathbf{P}_{\bm{r}} \bm{=} \dfrac{\,\bm{n!}\,}{\,\bm{(n-r)}\bm{!}\,}
        \label{eq:順列の総数②}
    \end{align}
\end{theorembox}
\vspace{1em}

Theorem\ref{thm:順列の総数②}が $r = 0, r = n$ のときも成り立つように, ${}_{\bm{n}}\mathbf{P}_{\bm{0}} \bm{= 1}$ , $\bm{0! = 1}$ と定めることとする.

\newpage

\begin{definitionbox}[def:円順列]{\textbf{円順列}}
    \label{def:円順列}
    いくつかのものを円形に並べる順列を \textbf{円順列} という.

    円順列では, \uwave{回転して並びが同じになるものは同じ並び方とみなす}.
\end{definitionbox}

\begin{theorembox}[thm:円順列の総数]{\textbf{円順列の総数}}
    \label{thm:円順列の総数}
    異なる $n$ 個のものの円順列の総数について, 次のことがいえる.

    \begin{align}
        \dfrac{\,_{\bm{n}}\mathbf{P}_{\bm{n}}\,}{\,\bm{n}\,} \bm{= (n-1)!}
        \label{eq:円順列の総数}
    \end{align}
\end{theorembox}

% じゅず順列
\vspace{1em}

ここまでは, 異なるものだけを並べる順列を考えてきた. ここでは, 同じものを繰り返し使うことを許した場合の順列を考えてみよう.

% 重複順列
\begin{definitionbox}[def:重複順列]{\textbf{重複順列}}
    \label{def:重複順列}
    一般に, \uwave{異なる $n$ 種類のものから重複を許して $r$ 個取って} 並べる順列を
    \bm{$n$} \textbf{個から} \bm{$r$} \textbf{個取る重複順列} という.

    重複順列では, $r \leqq n$ とは限らず, $r > n$ であってもよい.
\end{definitionbox}

\begin{theorembox}[thm:重複順列の総数]{\textbf{重複順列の総数}}
    \label{thm:重複順列の総数}
    重複順列の総数について, 次のことがいえる.

    \begin{align}
        n \text{個から} r \text{個取る重複順列の総数は} \qquad \bm{n^r}
        \label{eq:重複順列の総数}
    \end{align}
\end{theorembox}
\newpage

\subsection{組合せ}

% 組合わせ
\begin{definitionbox}[def:組合せ]{\textbf{組合せ}}
    \label{def:組合せ}
    一般に, \uwave{異なる $n$ 個のものから異なる $r$ 個を取り出して} 作る組合せを
    \begin{center}
        $\bm{n}$\textbf{個から}$\bm{r}$\textbf{個取る組合せ}
    \end{center}
    といい, その総数を $_{\bm{n}}\mathbf{C}_{\bm{r}}$ $^{*}$ で表す. ただし, $r \leqq n$ とする.
\end{definitionbox}
\vspace{1em}

ここで, Permutation と Combination の違いを確認しておこう.

まずは, 日本語の意味の違いから. Permutation は「順列」, Combination は「組合せ」である.

次に, 数学的な視点から, Permutation と Combination の違いを考えてみよう.















\vspace{1em}

\begin{theorembox}[thm:組合せの総数]{\textbf{組合せの総数 $_{\bm{n}}\mathbf{C}_{\bm{r}}$}}
    $n$ 個から $r$ 個取る組合せの総数 $_{n}\mathrm{C}_{r}$ は次の式で表される.

    \begin{align}
        {}_{\bm{n}}\mathrm{C}_{\bm{r}} \bm{=} \dfrac{\,{}_{\bm{n}}\mathrm{P}_{\bm{r}}\,}{\,\bm{r!}\,} \bm{=}
        \dfrac{\,\bm{n(n-1)(n-2)\cdots\cdots(n-r+1)}\,}{\,\bm{r(r-1)(r-2)\cdots\cdots \cdot 3 \cdot 2 \cdot 1}\,}
        \label{eq:組合せの総数}
    \end{align}
\end{theorembox}
\vspace{1em}

Theorem\ref{eq:組合せの総数}において, 特に ${}_n\mathrm{C}_1 = n$ , ${}_n\mathrm{C}_n = 1$ である.

また, 式\ref{eq:順列の総数②}より, ${}_{\bm{n}}\mathbf{C}_{\bm{r}} \bm{=} \dfrac{\,\bm{n!}\,}{\,\bm{r!(n-r)!}\,}$
と表すこともできる. ただし, ${}_{\bm{n}}\mathbf{C}_{\bm{0}} \bm{=} 1$ と定めることとする.
\vspace{1em}

\begin{theorembox}[thm:]{\textbf{$_{\bm{n}}\mathbf{C}_{\bm{r}}$ の性質}}
    一般に, $n$ 個から $r$ 個取る組合せの総数は, $n$ 個から $(n-r)$ 個取る組合わせの総数に等しい.
    すなわち, 次の等式が成り立つ.
    \begin{align}
        {}_{\bm{n}}\mathbf{C}_{\bm{r}} \bm{=} {}_{\bm{n}}\mathbf{C}_{\bm{n-r}}
    \end{align}
\end{theorembox}

\begin{theorembox}[thm:同じものを含む順列の総数]{\textbf{同じものを含む順列の総数}}
    $a$ が $p$ 個, $b$ が $q$ 個, $c$ が $r$ 個あるとき, それら全部を1列に並べる順列の総数は

    \begin{align}
        {}_{n}\mathrm{C}_{p} \times {}_{n-p}\mathrm{C}_{q} = \dfrac{\,\bm{n!}\,}{\,\bm{p!\;q!\;r!}\,} \qquad \qquad
        \text{ただし} \qquad p + q + r = n
        \label{eq:同じものを含む順列の総数}
    \end{align}
\end{theorembox}
\vspace{1em}

$r = 0$ のときは, ${}_{n-p}\mathrm{C}_{q} = 1$ であり, 順列の総数は $\dfrac{\,n!\,}{\,p!\;q!\,}$ である.
