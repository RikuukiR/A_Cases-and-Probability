% 第2節 確率
\section{確率}
\subsection{事象と確率}

% 確率
    \begin{definitionbox}[def:確率]{\textbf{確率}}
        ある事柄が起こることが期待される程度を表す数値を \textbf{確率} という.
    \end{definitionbox}

    \begin{definitionbox}[def:試行と事象]{\textbf{試行と事象}}
        「さいころを投げる」とか「くじを引く」などのように, 同じ条件のもとで繰り返すことができ,
        その結果が偶然によって決まる実験や観測を \textbf{試行} という.

        また, 試行の結果として起こる事柄を \textbf{事象} という.
    \end{definitionbox}

    \begin{definitionbox}[def:全事象と根元事象]{\textbf{全事象と根元事象}}
        1つの試行において, 起こりうる結果全体を集合 $U$ で表すとき, その試行におけるどの事象も,
        $U$ の部分集合で表すことができる. $U$ 自身で表される事象を \textbf{全事象} ,
        $U$ のただ1つの要素からなる集合で表される事象を \textbf{根元事象} という.
    \end{definitionbox}

    \begin{definitionbox}[def:同様に確からしい]{\textbf{同様に確からしい}}
        1つの試行において, ある事象 $A$ の起こる確率を $P(A)$ で表す. $^{*}$

        ある試行において, どの根元事象が起こることも同程度に期待できるとき, これらの根元事象は \textbf{同様に確からしい} という.$^{**}$
    \end{definitionbox}

    \begin{theorembox}[thm:事象$A$の起こる確率]{\textbf{事象$A$の起こる確率}}
        同様に確からしい試行において, 起こりうるすべての場合の数を $N$ , 事象 $A$ の起こる場合の数を $a$ とするとき,
        事象 $A$ の起こる確率 $\bm{P(A)}$ は次の式で表される.

        \begin{align}
            \bm{P(A)} \bm{=} \dfrac{\text{\textbf{事象 $A$ の起こる場合の数}}}{\text{\textbf{起こりうるすべての場合の数}}} \bm{=} \dfrac{\,\bm{a}\,}{\,\bm{N}\,}
            \label{eq:事象$A$の起こる確率}
        \end{align}
    \end{theorembox}

    \vfill

    \noindent
    \rule{\textwidth}{0.4pt}

    \noindent
    $^{*}$ $P(A)$ の $P$ は,  確率を意味するProbability の頭文字である. また, 事象$A$の起こる確率を単に \textbf{事象 $A$ の確率} ということもある.

    \noindent
    $^{**}$ ここから取り上げる試行では, \uwave{全事象 $U$ におけるすべての根元事象は同様に確からしいものとする}.
\newpage

% 例題8
\begin{example}[【教p.47 例題8】]
    A, B の2人を含む6人のリレー選手がいる. 走る順番をくじで決めるとき, A が1番目, B が6番目になる確率を求めよ.
\end{example}
\vspace{-1em}%  例題環境の最後の空行を減らす

\begin{proof}\mbox{}\\
    まず, 起こりうるすべての場合の数を考える.

    6人全員の並び順は, $\answermath{6!}$ 通りある.

    次に, 求めたい場合の数を考える.

    A が1番目, B が6番目のとき, A, B 以外の4人の並び順は, $\answermath{4!}$ 通りある.

    よって, 求める確率は

    \begin{alignanswer*}
        \dfrac{\answermath{4!}}{\answermath{6!}} = \dfrac{1}{30}
    \end{alignanswer*}
    \vspace{-4\baselineskip}%  証明環境の最後の空行を減らす
\end{proof}
\vspace{-1em}%  証明環境の最後の空行を減らす

% 例題9
\begin{example}[【教p.47 例題9】]
    赤玉 $4$ 個, 白玉 $5$ 個の入った袋から, $2$ の玉を同時に取り出すとき, 赤玉 $1$ 個, 白玉 $1$ 個が出る確率を求めよ.
\end{example}
\vspace{-1em}%  例題環境の最後の空行を減らす

\begin{proof}\mbox{}\\
    まず, 起こりうるすべての場合の数を考える.
    全部の $9$ 個から $2$ 個取る組合せは, $\answermath{{}_{9}\mathrm{C}_{2}}$ 通りある.

    次に, 求めたい場合の数を考える.

    赤玉 $4$ 個から $1$ 個, 白玉 $5$ 個から $1$ 個取る組合せは,
    $\answermath{{}_{4}\mathrm{C}_{1} \times {}_{5}\mathrm{C}_{1}}$ 通りある.

    よって, 求める確率は

    \begin{alignanswer*}
        \dfrac{\answermath{{}_{4}\mathrm{C}_{1} \times {}_{5}\mathrm{C}_{1}}}{\answermath{{}_{9}\mathrm{C}_{2}}}
        = 4 \times 5 \times \dfrac{2 \cdot 1}{9 \cdot 8} = \dfrac{5}{9}
    \end{alignanswer*}
    \vspace{-4\baselineskip}%  証明環境の最後の空行を減らす
\end{proof}
\newpage

% 確率の基本性質
\subsection{確率の基本性質}

確率の基本性質を議論する前に, 事象 $A$, $B$ に対して, 次のような事象を定義しておく.

    \begin{definitionbox}[def:積事象と和事象]{\textbf{積事象と和事象}}
        事象 $A$ と $B$ がともに起こる事象を $A$ , $B$ の \textbf{積事象} といい, $\bm{A \cap B}$ で表す.

        一方, 事象 $A$ または $B$ が起こる事象を $A$ と $B$ の \textbf{和事象} といい, $\bm{A \cup B}$ で表す.
    \end{definitionbox}

    \begin{definitionbox}[def:排反事象]{\textbf{排反事象}}
        2つの事象 $A$ , $B$ が決して同時に起こらないとき,
        \[A, B \text{は互いに} \textbf{排反} \text{である}\]
        または
        \[A, B \text{は互いに} \textbf{排反事象} \text{である}\]
        という.
    \end{definitionbox}

    \begin{definitionbox}[def:空事象]{\textbf{空事象}}
        $A$ , $B$ が互いに排反であることは, $A \cap B = \emptyset$ であることと同値である.

        空集合 $\emptyset$ で表される事象を \textbf{空事象} という. 空事象は決して起こらない事象である.
    \end{definitionbox}

    \begin{theorembox}[thm:確率の基本性質]{\textbf{確率の基本性質}}

    \begin{enumerate}
        \item どんな事象 $A$ についても
            \begin{align*}
                \phantom{\text{とくに、空事象 $\emptyset$ について} \qquad} &\bm{0 \leq P(A) \leq 1} \\
                \text{とくに、空事象 $\emptyset$ について} \qquad &\bm{P(\emptyset) = 0} \\
                \text{全事象 $U$ について} \qquad &\bm{P(U) = 1}
            \end{align*}

        \item \uwave{事象 $A$, $B$ が互いに排反であるとき}
            \begin{align}
                \bm{P(A \cup B) = P(A) + P(B)}
                \label{eq:排反事象の確率}
            \end{align}
        \end{enumerate}
    \end{theorembox}
    \vspace{1em}

    Theorem\ref{thm:確率の基本性質} の(2)を, 確率の \textbf{加法定理} という.

    3つ以上の事象については, どの2つの事象も互いに排反であるとき, これらは 互いに排反である という.
    3つ以上の排反な事象についても, 2つの場合の加法定理と同様なことが成り立つ.
\newpage

% 例題10
\begin{example}[【教p.51 例題10】]
    赤玉4個, 白玉5個の入った袋から, 2個の玉を同時に取り出すとき, 2個が同じ色である確率を求めよ.
\end{example}

\begin{proof}\mbox{}\\
    まず, 起こりうるすべての場合の数を考える.

    全部の $9$ 個から $2$ 個取る組合せは, $\answermath{{}_{9}\mathrm{C}_{2}}$ 通りある.

    次に, 求めたい場合の数を考える.

    ここで, 「2個が同じ色である」という事象は,
    \[「2個とも赤玉である」\text{という事象を} A,\]
    \[「2個とも白玉である」\text{という事象を} B\]
    の和事象 $A \cup B$ である.

    $A$, $B$ は互いに排反であるから, \answertext{確率の加法定理}より

    \begin{alignanswer*}
        P(A \cup B) = P(A) + P(B)
        = \dfrac{{}_{4}\mathrm{C}_{2}}{{}_{9}\mathrm{C}_{2}} + \dfrac{{}_{5}\mathrm{C}_{2}}{{}_{9}\mathrm{C}_{2}}
        = \dfrac{6}{36} + \dfrac{10}{36} = \dfrac{4}{9}
    \end{alignanswer*}
\end{proof}
\vspace{1em}

% 余事象
\begin{definitionbox}[def:余事象]{\textbf{余事象}}
    全事象を $U$ とする. 事象 $A$ に対して, 「$A$ が起こらない」という事象を $A$ の \textbf{余事象} といい, $\bm{\overline{A}}$ で表す.
    $A, \overline{A}$ は互いに排反である.
\end{definitionbox}

\begin{theorembox}[thm:余事象と確率]{\textbf{余事象と確率}}
    事象 $A$ に対して, $A$ が起こらない事象を $A$ の \textbf{余事象} といい, $\overline{A}$ で表す.

    \begin{align}
        \bm{P(\overline{A}) = 1 - P(A)}
        \label{eq:余事象と確率}
    \end{align}
\end{theorembox}
\newpage

% 例16
\begin{example}[【教p.52 例16】]
    1から100までの番号札100枚のから1枚を引くとき, 5の倍数でない番号を引く確率を求めよ.
\end{example}
\vspace{-1em}%  例題環境後の空行を減らす

\begin{proof}\mbox{}\\
    まず, 起こりうるすべての場合の数を考える.

    $100$ 枚から $1$ 枚引くため, $\answermath{{}_{100}\mathrm{C}_{1}}$ 通りある.

    次に, 求めたい場合の数を考える.

    引いた札の番号が「5の倍数でない」という事象は, 「\answertext{5の倍数である}」という事象の余事象である.

    5の倍数の番号を引く確率は, $\answermath{\dfrac{20}{100}}$

    よって, 求める確率は

    \begin{alignanswer*}
        1 - \dfrac{20}{100} = 1 - \dfrac{1}{5} = \dfrac{4}{5}
    \end{alignanswer*}
    \vspace{-4\baselineskip}%  この証明環境の空行を減らす
\end{proof}
\vspace{-0.5em}%  証明環境と例題環境間の空行を減らす

% 応用例題10
\begin{example}[【教p.52 応用例題10】]
    1から9までの番号札9枚のから4枚を同時に引くとき, 少なくとも1枚が偶数の番号である確率を求めよ.
\end{example}
\vspace{-1em}%  例題環境後の空行を減らす

\begin{proof}\mbox{}\\
    まず, 起こりうるすべての場合の数を考える.

    $9$ 枚から $4$ 枚引くため, $\answermath{{}_{9}\mathrm{C}_{4}}$ 通りある.

    次に, 求めたい場合の数を考える.

    「少なくとも1枚が偶数」という事象は, 「偶数が1枚もない」すなわち「\answertext{4枚とも奇数}」という事象の余事象である.

    奇数の札は5枚ある.

    よって, 4枚とも奇数である確率は, $\answermath{\dfrac{{}_{5}\mathrm{C}_{4}}{{}_{9}\mathrm{C}_{4}} = \dfrac{5}{126}}$

    よって, 求める確率は, この余事象の確率であるから

    \begin{alignanswer*}
        1 - \dfrac{5}{126} = \dfrac{121}{126}
    \end{alignanswer*}
    \vspace{-4\baselineskip}%  この証明環境の空行を減らす
\end{proof}
\newpage

% 一般の和事象の確率

ここまでは, 互いに排反な事象についての確率を考えてきた.
ここからは, 2つに事象 $A, B$ が互いに排反でないとき, 和事象の確率 $P(A \cup B)$ について考えてみよう.

\begin{theorembox}[thm:一般の和事象の確率]{\textbf{一般の和事象の確率}}
    全事象を $U$ とする. 2つの事象 $A, B$ が互いに排反でないとき, 和事象の確率 $P(A \cup B)$ は次の式で表される.

    \begin{align}
        \bm{P(A \cup B) = P(A) + P(B) - P(A \cap B)}
        \label{eq:一般の和事象の確率}
    \end{align}
\end{theorembox}

% 例17
\begin{example}[【教p.53 例17】]
    1から30までの30枚の番号札から1枚引くとき, その番号が2の倍数または3の倍数である確率を求めよ.
\end{example}

\begin{proof}\mbox{}\\
    まず, 起こりうるすべての場合の数を考える.

    $30$ 枚から $1$ 枚引くため, $\answermath{{}_{30}\mathrm{C}_{1}}$ 通りある.

    次に, 求めたい場合の数を考える.

    引いた札の番号が「2の倍数である」という事象を $A$ , 「3の倍数である」という事象を $B$ とすると,
    求める確率は $\answermath{P(A \cup B)}$ である.

    ここで,
    \begin{align*}
        A &= \{2 \cdot 1,\;\; 2 \cdot 2,\;\; 2 \cdot 3,\;\; \cdots \cdots,\;\; 2 \cdot 15\}\\
        B &= \{3 \cdot 1,\;\; 3 \cdot 2,\;\; 3 \cdot 3,\;\; \cdots \cdots,\;\; 3 \cdot 10\}\\
        A \cap B &= \{6 \cdot 1,\;\; 6 \cdot 2,\;\; 6 \cdot 3,\;\; \cdots \cdots,\;\; 6 \cdot 5\}
    \end{align*}
    であるから, \qquad $n(A) = \answertext{15},\; n(B) = \answertext{10},\; n(A \cap B) = \answertext{5}$ である.

    よって, 番号が2の倍数または3の倍数である確率は

    \begin{alignanswer*}
        P(A \cup B) = P(A) + P(B) - P(A \cap B)
        = \dfrac{15}{30} + \dfrac{10}{30} - \dfrac{5}{30}
        = \dfrac{20}{30} = \dfrac{2}{3}
    \end{alignanswer*}
\end{proof}
\newpage

\subsection{独立な試行と確率}

A, Bの2人がさいころを投げるとする. このとき, Aがさいころを投げる試行と, Bがさいころを投げる試行では,
それぞれの結果は互いに影響を与えない.

% 独立な試行
\begin{definitionbox}[def:独立な試行]{\textbf{独立な試行}}
    いくつかの試行において, どの試行の結果も他の試行の結果に影響を与えないとき,
    これらの試行は \textbf{独立} であるという.
\end{definitionbox}

一般に, 独立な2つの試行における事象の確率について, 次のことが成り立つ.

    \begin{theorembox}[thm:独立な試行の確率]{\textbf{独立な試行の確率}}
        2つの試行 $S$ と $T$ が独立であるとき, $S$ で事象 $A$ が起こり,かつ $T$ で事象 $B$ が起こる確率 $p$
        は,\quad $P(A)$ と $P(B)$ の積に等しい.

        すなわち,
        \begin{align}
            \bm{p = P(A) \times P(B)}
            \label{eq:独立な試行の確率}
        \end{align}
    \end{theorembox}

    独立な3つ以上の試行についても, Theorem\ref{thm:独立な試行の確率} と同様なことが成り立つ.

% 例18
\begin{exercise}[【教p.55 例18】]
    2枚の硬貨と1個のさいころを投げるとき, 次の確率を求めよ.
    \begin{enumerate}
        \item 硬貨は2枚とも表が出て, さいころは偶数の目が出る.
        \item 硬貨は1枚だけが表が出て, さいころは2以下の目が出る.
    \end{enumerate}
\end{exercise}
\vspace{-1em}%  練習問題環境の最後の空行を減らす

\begin{proof}\mbox{}\\
    (1) 「硬貨を投げる」という試行と, 「さいころを投げる」という試行は独立である.

    硬貨が2枚とも表が出る確率は, $\answermath{\dfrac{1}{2} \times \dfrac{1}{2} = \dfrac{1}{4}}$

    さいころが偶数の目が出る確率は, $\answermath{\dfrac{1}{2}}$

    よって, 求める確率は

    \begin{alignanswer*}
        \dfrac{1}{2} \times \dfrac{1}{2} \times \dfrac{1}{2} = \dfrac{1}{8}
    \end{alignanswer*}
    \vspace{-4\baselineskip}%  証明環境の最後の空行を減らす
\end{proof}
\newpage

% 例題11
\begin{example}[【教p.56 例題11】]
    Aの袋には赤玉3個と白玉2個, Bの袋には赤玉2個と白玉4個が入っている.
    A, Bの袋から1個ずつ玉を取り出すとき, 次の確率を求めよ.
    \begin{enumerate}
        \item ともに赤玉を取り出す確率
        \item 同じ色の玉を取り出す確率
    \end{enumerate}
\end{example}

\begin{proof}\mbox{}\\
    (1) 「Aの袋から玉を1個取り出す」という試行と, 「Bの袋から玉を1個取り出す」という試行は独立である.

    Aの袋から赤玉を取り出す確率は, $\answermath{\dfrac{3}{5}}$

    Bの袋から赤玉を取り出す確率は, $\answermath{\dfrac{2}{6}}$

    よって, 求める確率は

    \begin{alignanswer*}
        \dfrac{3}{5} \times \dfrac{2}{6}  = \dfrac{1}{5}
    \end{alignanswer*}
\end{proof}

\begin{proof}\mbox{}\\
    (2) 「Aの袋から玉を1個取り出す」という試行と, 「Bの袋から玉を1個取り出す」という試行は独立である.

    また, 「同じ色の玉を取り出す」という事象は, 「赤玉を取り出す」という場合と「白玉を取り出す」という場合がある.

    ともに赤玉を取り出す確率は, (1)より $\answermath{\dfrac{1}{5}}$

    ともに白玉を取り出す確率は, $\answermath{\dfrac{2}{5} \times \dfrac{4}{6} = \dfrac{4}{15}}$

    これらの事象は互いに排反であるから, 求める確率は

    \begin{alignanswer*}
        \dfrac{1}{5} + \dfrac{4}{15} = \dfrac{3}{15} + \dfrac{4}{15} = \dfrac{7}{15}
    \end{alignanswer*}
\end{proof}
\newpage

%反復試行の確率
\begin{definitionbox}[def:反復試行]{\textbf{反復試行}}
    同じ条件のもとでの試行の繰り返しを \textbf{反復試行} という.
    1つの試行を何回か繰り返すとき, これらの試行は独立である.
\end{definitionbox}
\vspace{-1em}%  定義環境の最後の空行を減らす

\begin{theorembox}[thm:反復試行の確率]{\textbf{反復試行の確率}}
    反復試行の確率について, 一般に次のことが成り立つ.

    1回の試行で事象 $A$ の起こる確率を $p$ とする.

    この試行を $n$ 回行う反復試行で, $A$ がちょうど $r$ 回起こる確率は

    \begin{align}
        {}_{\bm{n}}\mathrm{C}_{\bm{r}} \; \bm{p}^{\bm{r}} \; (\bm{1} - \bm{p})^{\bm{n -r}}
        \label{eq:反復試行の確率}
    \end{align}
\end{theorembox}

一般に, 正の数 $a$ に対して, $a ^0 = 1$ と定めることとする.

% 例題12
\begin{example}[【教p.58 例題12】]
    赤玉2個, 白玉4個の入った袋から, 玉を1個取り出し, 色を見てからもとにもどす. この試行を6回行うとき, 次の確率を求めよ.
    \begin{enumerate}
        \item 赤玉が5回以上出る確率
        \item 6回目に3度目の赤玉が出る確率
    \end{enumerate}
\end{example}
\vspace{-1.5em}%  例題環境の最後の空行を減らす

\begin{proof}\mbox{}\\
    (1) 1回の試行で赤玉が出る確率は, \quad $\answermath{\dfrac{1}{3}}$ である. また, 6回のうち赤玉が5回以上出るのは, 次の場合である.

    \[\answertext{赤玉がちょうど5回出る} \quad \quad \answertext{赤玉が6回出る}\]
    これらの事象は互いに排反であるから, 求める確率は

    \begin{alignanswer*}
        {}_{6}\mathrm{C}_{5} \left(\dfrac{1}{3}\right)^5 \left(\dfrac{2}{3}\right)^1 + \left(\dfrac{1}{3}\right)^6
        &= 6 \times \left(\dfrac{1}{3}\right)^5 \times \left(\dfrac{2}{3}\right) + \left(\dfrac{1}{3}\right)^6\\
        &= \dfrac{13}{729}
    \end{alignanswer*}
    \vspace{-6\baselineskip}%  証明環境の最後の空行を減らす
\end{proof}
\vspace{-2em}%  証明環境の最後の空行を減らす

\begin{proof}\mbox{}\\
    (2) 5回目までに \answertext{赤玉がちょうど2回} 出て, 6回目に3度目の赤玉が出る確率であるから

    \begin{alignanswer*}
        {}_{5}\mathrm{C}_{2} \left(\dfrac{1}{3}\right)^2 \left(\dfrac{2}{3}\right)^3 \times \dfrac{1}{3}
        &= 10 \times \left(\dfrac{1}{3}\right)^2 \times \left(\dfrac{2}{3}\right)^3 \times \left(\dfrac{1}{3}\right)\\
        &= \dfrac{80}{729}
    \end{alignanswer*}
    \vspace{-6\baselineskip}%  証明環境の最後の空行を減らす
\end{proof}
\newpage

% 応用例題11
\begin{example}[【教p.59 応用例題11】]
    数直線上を動く点 $P$ が原点の位置にある. 1枚の硬貨を投げて, 表が出たときは $P$ を正の向きに2だけ進め,
    裏が出たときは $P$ を負の向きに1だけ進める. 硬貨を6回投げ終わったとき, $P$ が原点にもどっている確率を求めよ.
\end{example}

\begin{techniquebox}[tech:81]{\textbf{反復試行は、(並び替えの数)$\times$(1個あたりの確率)で計算せよ!}}
    反復試行の確率は, \textbf{(並び替えの数) と (1個あたりの確率)を順に考える}ことで求めることができる.

    このように考えることによって, 事象が3つ以上の場合でも容易に確率を求めることができる.
\end{techniquebox}

\begin{proof}\mbox{}\\
    条件を満たす場合を考えて, 和の法則を使って確率を求めることもできる.
    しかし, この方法では, \textbf{漏れなく場合を分けること}と\textbf{時間がかかること}がデメリットである.

    したがって, 次のように文字を使用し, 一般化することによって求めることとする.
\vspace{1em}

    6回のうち, 表の回数を $r$ 回とすると, 裏の回数は $\answertext{6 - r}$ 回である.

    よって, 6回投げ終わったときの $P$ の座標は $\answertext{2r + (-1) (6 - r)}$ と表すことができる.

    硬貨を1回投げる時, 表が出る確率は \quad $\answermath{\dfrac{1}{2}}$ である.

    6回のうち, 表が $r$ 回出るとすると, 裏は $(6 - r)$ 回出るから6回で原点に戻るのは
    \[\answertext{2r + (-1) (6 - r) = 0}\]
    が成り立つときである.

    これを解くと \quad $\answermath{r = 2}$

    よって, $P$ が原点にもどっているのは6回のうち \uwave{表がちょうど2回} 出るときである.

    したがって, 求める確率は

    \begin{alignanswer*}
        {}_{6}\mathrm{C}_{2} \left(\dfrac{1}{2}\right)^2 \left(\dfrac{1}{2}\right)^4
        &= 15 \times \left(\dfrac{1}{2}\right)^2 \times \left(\dfrac{1}{2}\right)^4\\
        &= \dfrac{15}{64}
    \end{alignanswer*}
\end{proof}
\newpage

% 条件付き確率
\subsection{条件付き確率}

    \begin{definitionbox}{条件付き確率}{\textbf{条件付き確率}}
        一般に, 1つの試行における2つの事象 $A, B$ について, 事象 $A$ が起こったとして,
        そのときに事象 $B$ が起こる確率を, $A$ が起こったときの $B$ が起こる\textbf{条件付き確率}といい, $\bm{P_A(B)}$ で表す.

        ここで, 全事象を $U$ とする. 2つの事象 $A, B$ について, 条件付き確率 $P_A(B)$ は,

        \[\textbf{$A$ を全事象としたときに, 事象 $B$ が起こる確率}\]
        であり, 次の式で定義される. ただし, $n(A) \neq 0$ とする.

        \begin{align}
            \bm{P_A(B)} \bm{=} \dfrac{\,\bm{n(A \cap B)}\,}{\,\bm{n(A)}\,}
            \label{eq:条件付き確率1}
        \end{align}
    \end{definitionbox}

    式\ref{eq:条件付き確率1} の右辺の分母と分子を, それぞれ $n(U)$ で割ると, 次の等式が得られる.

    \begin{theorembox}[thm:条件付き確率]{\textbf{条件付き確率}}
        2つの事象 $A$, $B$ について, 次が成り立つ.

        \begin{align}
            \bm{P_A(B)} \bm{=} \dfrac{\,\bm{P(A \cap B)}\,}{\,\bm{P(A)}\,}
            \label{eq:条件付き確率2}
        \end{align}
    \end{theorembox}
    \vspace{-2em}%  定理環境の最後の空行を減らす

% 例題13
\begin{example}[【教p.61 例題13】]
    ある博物館の入館者のうち, 全体の20\%が高校生で, 全体の15\%が前売り券で入館した高校生である.
    入館した高校生の中から1人を選び出すとき, その人が前売り券で入館している確率を求めよ.
\end{example}
\vspace{-3em}%  例題環境の最後の空行を減らす

\begin{techniquebox}[tech:82]{\textbf{条件付き確率は、2つの確率を求めよ!}}
Theorem\ref{thm:条件付き確率} より, 「$A$ が起こる確率」と, 「$A$ と $B$ がともに起こる確率」の2つを求めるだけで, 条件付き確率を求めることができる.
すなわち, \textbf{条件付き確率の問題は確率2問分にすぎない} ということである.
\end{techniquebox}
\vspace{-1em}%  テクニックボックスの最後の空行を減らす

\begin{proof}\mbox{}\\
    入館者全体から選び出された1人が高校生であるという事象を $A$ , 前売り券で入館しているという事象を $B$ とすると

    \[P(A) = \answermath{\dfrac{20}{100}} , \qquad P(A \cap B) = \answermath{\dfrac{15}{100}}\]
    よって、求める確率は

    \begin{alignanswer*}
        P_A(B) = \dfrac{P(A \cap B)}{P(A)} = \dfrac{\dfrac{15}{100}}{\dfrac{20}{100}} = \dfrac{15}{20} = \dfrac{3}{4}
    \end{alignanswer*}
    \vspace{-7\baselineskip}%  証明環境の最後の空行を減らす
\end{proof}
\newpage

% 練習51
\begin{exercise}[【教p.61 練習51】]
    大人と子どもの人数の比が 3 : 2 であるグループに, ある提案をしたところ, 子供で賛成した人数は全体の15\%であった.
    このグループの子どもの中から1人選び出すとき, その人が提案に賛成である確率を求めよ.
\end{exercise}
\vspace{-1em}%  練習問題環境の最後の空行を減らす

\begin{proof}\mbox{}\\
    \answertext{グループから選び出された1人が子どもである} という事象を $A$ , \answertext{提案に賛成である} という事象を $B$ とすると

    \[P(A) = \answermath{\dfrac{2}{5}} , \qquad P(A \cap B) = \answermath{\dfrac{15}{100}}\]
    よって、求める確率は

    \begin{alignanswer*}
        P_A(B) = \dfrac{P(A \cap B)}{P(A)} &= \dfrac{\dfrac{15}{100}}{\dfrac{2}{5}}\\
        &= \dfrac{15}{100} \times \dfrac{5}{2} = \dfrac{3}{8}
    \end{alignanswer*}
    \vspace{-6\baselineskip}%  証明環境の最後の空行を減らす
\end{proof}
\vspace{-2em}%  証明環境と練習問題環境間の空行を減らす

% MASTERKEY
\begin{exercise}[【MASTERKEYp.127 例題】]
    ある製品の 50\% は工場 X, 30\% は工場 Y, 20\% は工場 Z で製造されており,
    工場X, Y, Z で製造された製品にはそれぞれ 5\%, 3\%, 2\% の確率で不良品が含まれる.

    この製品の中から無作為に1つを選んで調べたところ不良品であったとき, それが工場 X の製品である確率を求めよ.
\end{exercise}
\vspace{-1.5em}%  練習問題環境後の空行を減らす

\begin{proof}\mbox{}\\
    \answertext{不良品である} という事象を $A$ , \answertext{工場 X の製品である} という事象を $B$ とすると

    \begin{alignanswer*}
        P(A) &= \answermath{\dfrac{50}{100} \cdot \dfrac{5}{100} + \dfrac{30}{100} \cdot \dfrac{3}{100} + \dfrac{20}{100} \cdot \dfrac{2}{100}}
        = \dfrac{380}{10000} \\
        P(A \cap B) &= \answermath{\dfrac{50}{100} \cdot \dfrac{5}{100}} = \dfrac{250}{10000} \\
    \end{alignanswer*}
    \vspace{-0.5em}%  alignanswer*環境間の空行を減らす

    よって、求める確率は
    \begin{alignanswer*}
        P_A(B) &= \dfrac{P(A \cap B)}{P(A)} = \dfrac{\dfrac{250}{10000}}{\dfrac{380}{10000}} \\
        &= \dfrac{250}{380} = \dfrac{25}{38}
    \end{alignanswer*}
    \vspace{-5\baselineskip}%  証明環境の空行を減らす
\end{proof}
\newpage

式\ref{eq:条件付き確率2} において, 分母を払って得られる等式から, 次の確率の \textbf{乗法定理} が成り立つ.

% 確率の乗法定理
\begin{theorembox}[thm:確率の乗法定理]{\textbf{確率の乗法定理}}
    2つの事象 $A$, $B$ がともに起こる確率 $P(A \cap B)$ は, 次の式で表される.

    \begin{align}
        \bm{P(A \cap B) = P(A) P_A(B)}
        \label{eq:確率の乗法定理}
    \end{align}
\end{theorembox}

3つ以上の事象の場合についても、Theorem\ref{thm:確率の乗法定理} と同様なことが成り立つ.

% 例22
\begin{exercise}[【教p.62 例22】]
    当たりくじを4本含む10本くじを, A, B の2人がこの順に1本ずつ引く. ただし, 引いたくじはもとにもどさない.

    この試行において, A, B の2人とも当たる確率を求めよ.
\end{exercise}

\begin{proof}\mbox{}\\
    \answertext{A が当たる} という事象を $A$ , \answertext{B が当たる} という事象を $B$

    とすると, 求める確率は $P(A \cap B)$ は, 乗法定理により

    \[P(A \cap B) = \answermath{ P(A) P_A(B)}\]

    A が当たったときに, 残りのくじは9本で, 当たりくじは3本を含むから, 条件付き確率 $P_A(B)$ は

    \[P_A(B) = \answermath{\dfrac{3}{9}}\]

    よって, 求める確率は
    \[P(A \cap B) = P(A) P_A(B) = \answermath{\dfrac{4}{10} \times \dfrac{1}{3} = \dfrac{2}{15}}\]
\end{proof}

% 練習52
\begin{exercise}[【教p.62 練習52】]
    例22において, 次の確率を求めよ.
    \begin{enumerate}
        \item A が当たり, B がはずれる確率
        \item A がはずれ, B が当たる確率
        \item 2人ともはずれる確率
    \end{enumerate}
\end{exercise}
\vspace{-1em}%  練習問題環境後の空行を減らす

\begin{proof}\mbox{}\\
    (1) \answertext{A が当たる} という事象を $A$ , \answertext{B がはずれる} という事象を $\overline{B}$

    とすると, 求める確率は $P(A \cap \overline{B})$ は, 乗法定理により

    \[P(A \cap \overline{B}) = \answermath{ P(A) P_A(\overline{B})}\]

    A が当たったときに, 残りのくじは9本で, はずれくじは6本を含むから, 条件付き確率 $P_A(\overline{B})$ は

    \[P_A(\overline{B}) = \answermath{\dfrac{6}{9}}\]

    よって, 求める確率は

    \[P(A \cap \overline{B}) = P(A) P_A(\overline{B}) = \answermath{\dfrac{4}{10} \times \dfrac{6}{9} = \dfrac{4}{15}}\]
    \vspace{-5\baselineskip}%  証明環境の空行を減らす
\end{proof}
\vspace{-1.5em}%  証明環境とproblemsolution環境間の空行を減らす

\begin{problemsolution}[8cm]
    (2) \answertext{A がはずれる} という事象を $\overline{A}$ , \answertext{B が当たる} という事象を $B$

    とすると, 求める確率は $P(\overline{A} \cap B)$ は, 乗法定理により

    \[P(\overline{A} \cap B) = \answermath{ P(\overline{A}) P_{\comp{A}}(B)}\]

    A がはずれたときに, 残りのくじは9本で, 当たりくじは4本を含むから, 条件付き確率 $P_{\comp{A}}(B)$ は

    \[P_{\comp{A}}(B) = \answermath{\dfrac{4}{9}}\]

    よって, 求める確率は
    \[P(\overline{A} \cap B) = P(\overline{A}) P_{\comp{A}}(B) = \answermath{\dfrac{6}{10} \times \dfrac{4}{9} = \dfrac{4}{15}}\]
    \vspace{-2\baselineskip}%  problemsolution環境の最後の空行を削除
\end{problemsolution}
\vspace{-7em}%  problemsolution環境間の空行を減らす

\begin{problemsolution}[8cm]
    (3) 求める確率は $P(\overline{A} \cap \overline{B})$ は, 乗法定理により

    \[P(\overline{A} \cap \overline{B}) = \answermath{ P(\overline{A}) P_{\comp{A}}(\overline{B})}\]

    A がはずれたときに, 残りのくじは9本で, はずれくじは5本を含むから, 条件付き確率 $P_{\comp{A}}(\overline{B})$ は

    \[P_{\comp{A}}(\overline{B}) = \answermath{\dfrac{5}{9}}\]

    よって, 求める確率は
    \[P(\overline{A} \cap \overline{B}) = P(\overline{A}) P_{\comp{A}}(\overline{B}) = \answermath{\dfrac{6}{10} \times \dfrac{5}{9} = \dfrac{30}{90} = \dfrac{1}{3}}\]
    \vspace{-2\baselineskip}%  problemsolution環境の最後の空行を削除
\end{problemsolution}
\vspace{-7em}%  problemsolution環境間の空行を減らす
\newpage

% 期待値
\subsection{期待値}

\begin{definitionbox}[def:期待値]{\textbf{期待値}}
    一般に, ある試行の結果に応じて, $x_1, x_2, \; \cdots\cdots, \; x_n$ のどれか1つの値をとる数量 $X$ があり,
    各値をとる確率が

    \[p_1, p_2, \; \cdots\cdots, \; p_n \qquad \text{ただし} \quad p_1 + p_2 + \; \cdots\cdots + \; p_n = 1\]

    であるとき, $x_1 p_1 + x_2 p_2 + \; \cdots\cdots + \; x_n p_n$ を数量 $X$ の \textbf{期待値} という.
\end{definitionbox}

\begin{theorembox}[thm:期待値]{\textbf{期待値}}
    \begin{minipage}[t]{0.5\textwidth}
        $X$ の取りうる値と確率が右の表のようなとき, $X$ の期待値は, 次の式で与えられる.
    \end{minipage}
    \hfill
    \begin{minipage}[t]{0.45\textwidth}
        \begin{center}
        \begin{tabular}{c||cccc|c}
            $X$ & $x_1$ & $x_2$ & $\cdots$ & $x_n$ & 計 \\
            \hline
            確率 & $p_1$ & $p_2$ & $\cdots$ & $p_n$ & $1$ \\
        \end{tabular}
        \end{center}
    \end{minipage}

    \vspace{0.5em}
    \[\bm{x_1 p_1 + x_2 p_2 + \; \cdots\cdots + \; x_n p_n}\]
    \label{eq:期待値}
\end{theorembox}
\vspace{-2em}% 定理環境の最後の空行を減らす

% 練習56
\begin{exercise}[【教p.68 練習56】]
    5枚の硬貨を同時に投げて, 表の出た枚数が 5, 4, 3 の場合に, それぞれ得点 40, 16, 4 を得るが,
    それ以外の場合には得点は得られないとする. 得点の期待値を求めよ.
\end{exercise}

\begin{proof}\mbox{}\\
    1枚の硬貨を投げるとき, 表の出る確率は $\answermath{\dfrac{1}{2}}$ , 裏の出る確率は $\answermath{\dfrac{1}{2}}$ である.

    5枚の硬貨を投げて5枚とも表が出る確率は

    \[\answermath{{}_5\mathrm{C}_5 \left(\dfrac{1}{2}\right)^5 \left(\dfrac{1}{2}\right)^0 = \dfrac{1}{32}}\]

    5枚の硬貨を投げて4枚が表で1枚が裏が出る確率は

    \[\answermath{{}_5\mathrm{C}_4 \left(\dfrac{1}{2}\right)^4 \left(\dfrac{1}{2}\right)^1 = \dfrac{5}{32}}\]

    5枚の硬貨を投げて3枚が表で2枚が裏が出る確率は

    \[\answermath{{}_5\mathrm{C}_3 \left(\dfrac{1}{2}\right)^3 \left(\dfrac{1}{2}\right)^2 = \dfrac{10}{32}}\]
    したがって, 得点と確率の関係は次の表のようになる.

    \begin{center}
    \begin{tabular}{c||ccc|c}
        得点 & 40 & 16 & 4 & 計 \\
        \hline
        確率 & \quad $\answermath{\dfrac{1}{32}}$ \quad & \quad $\answermath{\dfrac{5}{32}}$ \quad & \quad $\answermath{\dfrac{10}{32}}$ \quad & $1$ \\
    \end{tabular}
    \end{center}

    よって、得点の期待値は

    \[\answermath{40 \times \dfrac{1}{32} + 16 \times \dfrac{5}{32} + 4 \times \dfrac{10}{32}
    = \dfrac{1}{32}\left(40 + 80 + 40\right) = \dfrac{160}{32} = 5}\]
    \vspace{-6\baselineskip}%  証明終了記号を前のページに押し込む
\end{proof}

% 応用例題12
    \begin{example}[【教p.69 応用例題12】]
        赤玉2個と白玉3が入った袋から, 3個の玉を同時に取り出し, 出た赤玉1個につき100円もらえるゲームがある.
        このゲームの参加料が150円のとき, このゲームに参加することは得であるといえるか.
    \end{example}

    \begin{problemsolution}[15cm]\mbox{}\\
        受け取る金額の期待値が参加料よりも大きいときに得であると判断する.
    \vspace{1em}

    出る赤玉の個数は,0, 1, 2 のいずれかである。

    赤玉が0個の確率は

    \[\dfrac{{}_3\mathrm{C}_0}{{}_5\mathrm{C}_3} = \dfrac{1}{10}\]

    赤玉が1個の確率は

    \[\dfrac{{}_2\mathrm{C}_1 \times {}_3\mathrm{C}_2}{{}_5\mathrm{C}_3} = \dfrac{6}{10}\]

    赤玉が2個の確率は

    \[\dfrac{{}_2\mathrm{C}_2 \times {}_3\mathrm{C}_1}{{}_5\mathrm{C}_3} = \dfrac{3}{10}\]

    よって, 受け取る金額を $X$ 円とすると, 次のような表ができる.

    \begin{center}
    \begin{tabular}{c||ccc|c}
        $X$ & 0 & 100 & 200 & 計 \\
        \hline
        確率 & \quad $\dfrac{1}{10}$ \quad & \quad $\dfrac{6}{10}$ \quad & \quad $\dfrac{3}{10}$ \quad & $1$ \\
    \end{tabular}
    \end{center}

    したがって, ゲームに参加したときの受け取る金額の期待値は

    \[0 \times \dfrac{1}{10} + 100 \times \dfrac{6}{10} + 200 \times \dfrac{3}{10}
    = \dfrac{1}{10}\left(0 + 600 + 600\right) = 120\]

    よって, 受け取る金額の期待値がゲームの参加料より小さいから, ゲームに参加することは得であるといえない.

    \end{problemsolution}
\newpage
