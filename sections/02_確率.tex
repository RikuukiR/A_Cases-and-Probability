% 第2節 確率
\section{確率}
\subsection{事象と確率}

    \begin{definitionbox}{}{\textbf{確率の定義}}
        試行において, 起こりうるすべての場合が $n$ 通りあり, そのどの場合が起こることも同様に確からしいとする.
        このとき, 事象 $A$ の起こる場合が $a$ 通りあるとすれば, 事象 $A$ の起こる\textbf{確率} $P(A)$ は次のように定義される.

        \begin{align}
            P(A) = \frac{a}{n}
            \label{eq:確率の定義}
        \end{align}
    \end{definitionbox}

\subsection{確率の基本性質}

    \begin{theorembox}[thm:確率の基本性質]{\textbf{確率の基本性質}}
        事象 $A$ の確率 $P(A)$ について, 次の性質が成り立つ.

        \begin{align}
            &0 \leq P(A) \leq 1
            \label{eq:確率の範囲}\\
            &P(U) = 1, \quad P(\emptyset) = 0
            \label{eq:全事象と空事象}\\
            &P(\bar{A}) = 1 - P(A)
            \label{eq:余事象の確率}
        \end{align}

        ここで, $U$ は全事象, $\emptyset$ は空事象, $\bar{A}$ は事象 $A$ の余事象を表す.
    \end{theorembox}

% \subsection{確率の加法定理}

    \begin{theorembox}[thm:確率の加法定理]{\textbf{確率の加法定理}}
        2つの事象 $A$, $B$ について, 次が成り立つ.

        \begin{align}
            P(A \cup B) = P(A) + P(B) - P(A \cap B)
            \label{eq:確率の加法定理}
        \end{align}

        特に, $A$ と $B$ が互いに排反である場合($A \cap B = \emptyset$), 次が成り立つ.

        \begin{align}
            P(A \cup B) = P(A) + P(B)
            \label{eq:排反事象の確率}
        \end{align}
    \end{theorembox}

\subsection{独立な試行と確率}

    \begin{theorembox}[thm:独立な試行の確率]{\textbf{独立な試行の確率}}
        2つの試行 $T_1$, $T_2$ が独立であるとき, 事象 $A$, $B$ について次が成り立つ.

        \begin{align}
            P(A \cap B) = P(A) \cdot P(B)
            \label{eq:独立な試行の確率}
        \end{align}
    \end{theorembox}

% \subsection{反復試行の確率}

    \begin{theorembox}[thm:反復試行の確率]{\textbf{反復試行の確率}}
        1回の試行で事象 $A$ が起こる確率を $p$ とする.
        この試行を $n$ 回繰り返すとき, 事象 $A$ がちょうど $r$ 回起こる確率は次の式で表される.

        \begin{align}
            {}_n\mathrm{C}_r \cdot p^r \cdot (1-p)^{n-r}
            \label{eq:反復試行の確率}
        \end{align}
    \end{theorembox}

\subsection{条件付き確率}

    \begin{definitionbox}{}{\textbf{条件付き確率}}
        事象 $A$ が起こったという条件のもとで事象 $B$ が起こる\textbf{条件付き確率} $P_A(B)$ は次のように定義される.

        \begin{align}
            P_A(B) = \frac{P(A \cap B)}{P(A)} \quad (P(A) > 0)
            \label{eq:条件付き確率}
        \end{align}
    \end{definitionbox}

    \begin{theorembox}[thm:確率の乗法定理]{\textbf{確率の乗法定理}}
        2つの事象 $A$, $B$ について, 次が成り立つ.

        \begin{align}
            P(A \cap B) = P(A) \cdot P_A(B)
            \label{eq:確率の乗法定理}
        \end{align}
    \end{theorembox}


    \subsection{期待値}

    \begin{definitionbox}{}{\textbf{条件付き確率}}
        事象 $A$ が起こったという条件のもとで事象 $B$ が起こる\textbf{条件付き確率} $P_A(B)$ は次のように定義される.

        \begin{align}
            P_A(B) = \frac{P(A \cap B)}{P(A)} \quad (P(A) > 0)
            % \label{eq:条件付き確率}
        \end{align}
    \end{definitionbox}

    \begin{theorembox}[thm:確率の乗法定理]{\textbf{確率の乗法定理}}
        2つの事象 $A$, $B$ について, 次が成り立つ.

        \begin{align}
            P(A \cap B) = P(A) \cdot P_A(B)
            % \label{eq:確率の乗法定理}
        \end{align}
    \end{theorembox}



