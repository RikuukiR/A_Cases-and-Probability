% 第2節 確率
\section{確率}
\subsection{事象と確率}

% 確率
    \begin{definitionbox}[def:確率]{\textbf{確率}}
        ある事柄が起こることが期待される程度を表す数値を \textbf{確率} という.
    \end{definitionbox}

    \begin{definitionbox}[def:試行と事象]{\textbf{試行と事象}}
        「さいころを投げる」とか「くじを引く」などのように, 同じ条件のもとで繰り返すことができ,
        その結果が偶然によって決まる実験や観測を \textbf{試行} という.

        また, 試行の結果として起こる事柄を \textbf{事象} という.
    \end{definitionbox}

    \begin{definitionbox}[def:全事象と根元事象]{\textbf{全事象と根元事象}}
        1つの試行において, 起こりうる結果全体を集合 $U$ で表すとき, その試行におけるどの事象も,
        $U$ の部分集合で表すことができる. $U$ 自身で表される事象を \textbf{全事象} ,
        $U$ のただ1つの要素からなる集合で表される事象を \textbf{根元事象} という.
    \end{definitionbox}

    \begin{definitionbox}[def:同様に確からしい]{\textbf{同様に確からしい}}
        1つの試行において, ある事象 $A$ の起こる確率を $P(A)$ で表す. $^{*}$

        ある試行において, どの根元事象が起こることも同程度に期待できるとき, これらの根元事象は \textbf{同様に確からしい} という.$^{**}$
    \end{definitionbox}

    \begin{theorembox}[thm:事象$A$の起こる確率]{\textbf{事象$A$の起こる確率}}
        同様に確からしい試行において, 起こりうるすべての場合の数を $N$ , 事象 $A$ の起こる場合の数を $a$ とするとき,
        事象 $A$ の起こる確率 $\bm{P(A)}$ は次の式で表される.

        \begin{align}
            \bm{P(A)} \bm{=} \dfrac{\text{\textbf{事象 $A$ の起こる場合の数}}}{\text{\textbf{起こりうるすべての場合の数}}} \bm{=} \dfrac{\,\bm{a}\,}{\,\bm{N}\,}
            \label{eq:事象$A$の起こる確率}
        \end{align}
    \end{theorembox}

    \vfill

    \noindent
    \rule{\textwidth}{0.4pt}

    \noindent
    $^{*}$ $P(A)$ の $P$ は,  確率を意味するProbability の頭文字である. また, 事象$A$の起こる確率を単に \textbf{事象 $A$ の確率} ということもある.

    \noindent
    $^{**}$ ここから取り上げる試行では, \uwave{全事象 $U$ におけるすべての根元事象は同様に確からしいものとする}.
\newpage

% 例題8
\begin{example}[【教p.47 例題8】]
    A, B の2人を含む6人のリレー選手がいる。走る順番をくじで決めるとき、A が1番目、B が6番目になる確率を求めよ。
\end{example}

\begin{proof}\mbox{}\\
    まず、起こりうるすべての場合の数を考える。

    6人全員の並び順は, $\answermath{6!}$ 通りある.

    次に、求めたい場合の数を考える。

    A が1番目、B が6番目のとき, A, B 以外の4人の並び順は, $\answermath{4!}$ 通りある.

    よって、求める確率は

    \begin{alignanswer*}
        \dfrac{\answermath{4!}}{\answermath{6!}} = \dfrac{1}{30}
    \end{alignanswer*}
\end{proof}

% 例題9
\begin{example}[【教p.47 例題9】]
    赤玉 $4$ 個, 白玉 $5$ 個の入った袋から、$2$ の玉を同時に取り出すとき、赤玉 $1$ 個、白玉 $1$ 個が出る確率を求めよ。
\end{example}

\begin{proof}\mbox{}\\
    まず、起こりうるすべての場合の数を考える。
    全部の $9$ 個から $2$ 個取る組合せは, $\answermath{{}_{9}\mathrm{C}_{2}}$ 通りある.

    次に、求めたい場合の数を考える。

    赤玉 $4$ 個から $1$ 個、白玉 $5$ 個から $1$ 個取る組合せは,
    $\answermath{{}_{4}\mathrm{C}_{1} \times {}_{5}\mathrm{C}_{1}}$ 通りある.

    よって、求める確率は

    \begin{alignanswer*}
        \dfrac{\answermath{{}_{4}\mathrm{C}_{1} \times {}_{5}\mathrm{C}_{1}}}{\answermath{{}_{9}\mathrm{C}_{2}}}
        = 4 \times 5 \times \dfrac{2 \cdot 1}{9 \cdot 8} = \dfrac{5}{9}
    \end{alignanswer*}
\end{proof}
\newpage

\subsection{確率の基本性質}

確率の基本性質を議論する前に、事象 $A$, $B$ に対して、次のような事象を定義しておく。

    \begin{definitionbox}[def:積事象と和事象]{\textbf{積事象と和事象}}
        事象 $A$ と $B$ がともに起こる事象を $A$ , $B$ の \textbf{積事象} といい, $\bm{A \cap B}$ で表す.

        一方, 事象 $A$ または $B$ が起こる事象を $A$ と $B$ の \textbf{和事象} といい, $\bm{A \cup B}$ で表す.
    \end{definitionbox}

    \begin{definitionbox}[def:排反事象]{\textbf{排反事象}}
        2つの事象 $A$ 、 $B$ が決して同時に起こらないとき、
        \[A, B \text{は互いに} \textbf{排反} \text{である}\]
        または
        \[A, B \text{は互いに} \textbf{排反事象} \text{である}\]
        という。
    \end{definitionbox}

    \begin{definitionbox}[def:空事象]{\textbf{空事象}}
        $A$ 、 $B$ が互いに排反であることは、$A \cap B = \emptyset$ であることと同値である。

        空集合 $\emptyset$ で表される事象を \textbf{空事象} という。空事象は決して起こらない事象である。
    \end{definitionbox}

    \begin{theorembox}[thm:確率の基本性質]{\textbf{確率の基本性質}}
        事象 $A$ の確率 $P(A)$ について, 次の性質が成り立つ.

        \begin{align}
            &0 \leq P(A) \leq 1
            \label{eq:確率の範囲}\\
            &P(U) = 1, \quad P(\emptyset) = 0
            \label{eq:全事象と空事象}\\
            &P(\bar{A}) = 1 - P(A)
            \label{eq:余事象の確率}
        \end{align}

        ここで, $U$ は全事象, $\emptyset$ は空事象, $\bar{A}$ は事象 $A$ の余事象を表す.
    \end{theorembox}

% \subsection{確率の加法定理}

    \begin{theorembox}[thm:確率の加法定理]{\textbf{確率の加法定理}}
        2つの事象 $A$, $B$ について, 次が成り立つ.

        \begin{align}
            P(A \cup B) = P(A) + P(B) - P(A \cap B)
            \label{eq:確率の加法定理}
        \end{align}

        特に, $A$ と $B$ が互いに排反である場合($A \cap B = \emptyset$), 次が成り立つ.

        \begin{align}
            P(A \cup B) = P(A) + P(B)
            \label{eq:排反事象の確率}
        \end{align}
    \end{theorembox}

\subsection{独立な試行と確率}

    \begin{theorembox}[thm:独立な試行の確率]{\textbf{独立な試行の確率}}
        2つの試行 $T_1$, $T_2$ が独立であるとき, 事象 $A$, $B$ について次が成り立つ.

        \begin{align}
            P(A \cap B) = P(A) \cdot P(B)
            \label{eq:独立な試行の確率}
        \end{align}
    \end{theorembox}

% \subsection{反復試行の確率}

    \begin{theorembox}[thm:反復試行の確率]{\textbf{反復試行の確率}}
        1回の試行で事象 $A$ が起こる確率を $p$ とする.
        この試行を $n$ 回繰り返すとき, 事象 $A$ がちょうど $r$ 回起こる確率は次の式で表される.

        \begin{align}
            {}_n\mathrm{C}_r \cdot p^r \cdot (1-p)^{n-r}
            \label{eq:反復試行の確率}
        \end{align}
    \end{theorembox}

\subsection{条件付き確率}

    \begin{definitionbox}{}{\textbf{条件付き確率}}
        事象 $A$ が起こったという条件のもとで事象 $B$ が起こる\textbf{条件付き確率} $P_A(B)$ は次のように定義される.

        \begin{align}
            P_A(B) = \frac{\,P(A \cap B)\,}{\,P(A)\,} \quad (P(A) > 0)
            \label{eq:条件付き確率}
        \end{align}
    \end{definitionbox}

    \begin{theorembox}[thm:確率の乗法定理]{\textbf{確率の乗法定理}}
        2つの事象 $A$, $B$ について, 次が成り立つ.

        \begin{align}
            P(A \cap B) = P(A) \cdot P_A(B)
            \label{eq:確率の乗法定理}
        \end{align}
    \end{theorembox}


    \subsection{期待値}

    \begin{definitionbox}{}{\textbf{条件付き確率}}
        事象 $A$ が起こったという条件のもとで事象 $B$ が起こる\textbf{条件付き確率} $P_A(B)$ は次のように定義される.

        \begin{align}
            P_A(B) = \frac{\,P(A \cap B)\,}{\,P(A)\,} \quad (P(A) > 0)
            % \label{eq:条件付き確率}
        \end{align}
    \end{definitionbox}

    \begin{theorembox}[thm:確率の乗法定理]{\textbf{確率の乗法定理}}
        2つの事象 $A$, $B$ について, 次が成り立つ.

        \begin{align}
            P(A \cap B) = P(A) \cdot P_A(B)
            % \label{eq:確率の乗法定理}
        \end{align}
    \end{theorembox}



