\documentclass[12pt,a4paper,titlepage]{jsreport}

\usepackage[margin=20truemm]{geometry}

\input{preamble.tex}

\setcounter{chapter}{-1}

\setcounter{tocdepth}{3}

\usepackage{enumerate}

\title{数学A}

\author{Riku Sugawara}

\date{}

\begin{document}

\maketitle

\tableofcontents

\clearpage

\chapter{はじめにと準備}

% \pagewiselinenumbers  % linenoパッケージが必要な場合有効化

数学という学問は定義の上で成り立っている.定義は正しく覚えて,内容をよく理解してほしい.\\

第1章「場合の数と確率」では,順列・組合せについて学んだ後,偶然性を伴う現象を数学的に扱う確率の考え方について学んでもらう.特に,場面に応じて適切な事象に分割して考えることで,系統的に確率を理解することを考察する.\\

第2章「図形の性質」では,三角形や円を含む図形に現れる線分の比について学んでもらう.また,作図の問題も考える.いろいろなゲームは,我々の使える手は限られているのが普通だが,作図の問題は,決められたルールで定規とコンパスだけを使うゲームである.\\

第3章「数学と人間の活動」では,数学の考え方と人間の活動との関りについて学んでもらう.最初に,自然数の素因数分解の性質を復習し,2つの自然数の最大公約数を求めるユークリッドの互除法,記数法などについて学んでもらう.次に,座標の考え方や,ゲーム・パズルの中の数学について考える.\\

\newpage

第1章に先立って,第1章『場合の数と確率』で必要となる数学Iの『集合』の内容を復習しておく.

\vskip\baselineskip

「1から10までの自然数の集まり」というと,その集まりの範囲がはっきりしている.これに対し,「大きい数の集まり」というと,その集まりの範囲が漠然としている.

数学では,範囲がはっきりしたものの集まりを集合といい,その集合を構成している1つ1つのものをその集合の要素という.改めて,

\begin{definitionbox}{}{\textbf{集合とその表し方}}
明確に定められた条件を満たす対象の集まりを \textbf{集合} という.また,集合を構成する個々の対象をその集合の \textbf{要素} という.
\end{definitionbox}

\vskip\baselineskip

\begin{definitionbox}{}{\textbf{集合と要素の関係を表す記号}}
集合 $X$ に対して $x$ が $X$ の要素であることを
\[x \in X \quad または \quad X \ni x\]
で表し, $x$\textbf{は} $X$ \textbf{の要素である} または $x$\textbf{は 集合} $X$ \textbf{に属する} という.

また,$y$ が $X$ の要素でないことを
\[y \notin X \quad または \quad X \not\ni y\]
と表す.
\end{definitionbox}

\vskip\baselineskip

「1から10までの自然数の集まり」と「大きい数の集まり」では,どちらが集合として認められるかは簡単だろう.言うまでもなく,前者である.

例えば,1から10までの自然数のうち,奇数全体の集合をPとすると,$P$は
\[1,3,5,7,9,\]
を要素とする集合である.この集合Pについては,次が成り立つ.
\[3 \in P, \quad 4 \notin P\]
ここで,集合の表し方について確認しておこう.

\begin{definitionbox}{}{\textbf{集合の表し方}}
集合を構成するすべての要素を書き並べ,書き並べた要素を \textbf{{ }(中括弧)} で囲み,その集合を表す.また,要素が複数存在する場合,要素と要素を \textbf{ ,(コンマ)} で区切り,それぞれの要素を区別できるようにする.この集合の表し方を \textbf{外延的記法} という.

一方, \textbf{要素} の代表を例えば $x$ で表し,{ }の中の縦線の右に,$x$ の満たす \textbf{条件} を書く方法もある.この集合の表し方を \textbf{内延的記法} という.
\end{definitionbox}

\vskip\baselineskip

今,$P$を1から10までの自然数のうち,奇数全体の集合とすると集合$P$は,外延的記法を用いて,
\[P = \{1 , 3 , 5 , 7 , 9\}\]
のように表せる.

一方,集合$P$を内延的記法で表すと,
\[P = \{\; x \;|\; x は1から10までの自然数\;\} =\{\; x \in \mathbb{ N }\;|\; 1 \leqq x \leqq 10 \;\}\]
となる.

また,$3$ は集合Pの要素であるが,$4$は集合Pの要素でない.つまり,$3 \in P, \quad 4 \notin P$であった.

\vskip\baselineskip

数学ではよく,数の集合を次のように表す.

\begin{align*}
\mathbb{ N } \quad &:\quad \textbf{自然数全体}からなる集合\\
\mathbb{ Z } \quad &:\quad \textbf{整数全体}からなる集合\\
\mathbb{ Q } \quad &:\quad \textbf{有理数全体}からなる集合\\
\mathbb{ R } \quad &:\quad \textbf{実数全体}からなる集合\\
\mathbb{ C } \quad &:\quad \textbf{複素数全体}からなる集合
\end{align*}

\begin{definitionbox}{}{\textbf{有限集合と無限集合}}
有限個の要素からなる集合を \textbf{有限集合} といい,無限に多くの要素からなる集合を \textbf{無限集合} という.
\end{definitionbox}

\vskip\baselineskip

上の例の$P$は有限集合であり,
\[Q = \{\;x\;|\;xは20以上の自然数\;\}\]
で定めた$Q$は無限集合である.

% 以下, 集合の定義や定理が続く...

\newpage

\chapter{場合の数と確率}

\section{場合の数}

% 第1節 場合の数
\section{場合の数}
\subsection{集合の要素の個数}
% 割愛

\subsection{場合の数}
% 割愛

\subsection{順列}
% 順列
\begin{definitionbox}[def:順列]{\textbf{順列}}
    \label{def:順列}
    いくつかのものを順に1列に並べるとき, その並びの1つ1つを \textbf{順列} という.

    一般に, \uwave{異なる $n$ 個のものから異なる $r$ 個を取り出して}並べる順列を
    \begin{center}
        $\bm{n}$\textbf{個から}$\bm{r}$\textbf{個取る順列}
    \end{center}
    といい, その総数を $_{\bm{n}}\mathbf{P}_{\bm{r}}$ $^{*}$ で表す.
\end{definitionbox}

\begin{theorembox}[thm:順列の総数①]{\textbf{順列の総数① $_{\bm{n}}\mathbf{P}_{\bm{r}}$}}
    \label{thm:順列の総数①}
    $n$ 個から $r$ 個取る順列の総数 $_{n}\mathrm{P}_{r}$ は次の式で表される.

    \begin{align}
        _{\bm{n}}\mathbf{P}_{\bm{r}} = \bm{n \cdot (n-1) \cdot (n-2) \cdots \cdots (n-r+1)}
        \label{eq:順列の総数①}
    \end{align}
\end{theorembox}
\vspace{1em}

順列の総数 ${}_n\mathrm{P}_r$ の式で, 特に $r = n$ のときは等式

\begin{align*}
    _{n}\mathrm{P}_{n} = n(n-1)(n-2) \cdots \cdots 3 \cdot 2 \cdot 1 \tag{*}
\end{align*}

が得られる.

% 階乗
\begin{definitionbox}[def:階乗]{\textbf{階乗}}
    \label{def:階乗}
    $(*)$ の右辺は, $1$ から $n$ までのすべての自然数の積である.

    これを $n$ の \textbf{階乗} といい, $\bm{n!}$ で表す.
\end{definitionbox}


\vfill

\noindent
\rule{\textwidth}{0.4pt}

\noindent
$^{*}$ ${}_n\mathrm{P}_r$ の P は 順列を意味するPermutation の頭文字である.\\

\newpage

\begin{theorembox}[thm:階乗]{\textbf{階乗}}
    \label{thm:階乗}
    Definition\ref{def:階乗} より, $n$ の階乗 $n!$ は次の式で表される.

    \begin{align}
        _{\bm{n}}\mathbf{P}_{\bm{r}} \bm{=} \bm{n!} \bm{=} \bm{n (n-1) (n-2) \cdots \cdots 3 \cdot 2 \cdot 1}
        \label{eq:階乗}
    \end{align}
\end{theorembox}
\vspace{1em}

Definition\ref{def:順列}より, 一般に, 次のことがいえる.

\[\textbf{異なる} \bm{n} \textbf{個のものすべてを並べる順列の総数は}\;\; \bm{n!} \;\;\textbf{である}\]

また, 順列の総数 ${}_n\mathrm{P}_r$ の式で, $r < n$ のときについて考える.
\vspace{1em}

% 証明


\begin{theorembox}[thm:順列の総数②]{\textbf{順列の総数② $_{\bm{n}}\mathbf{P}_{\bm{r}}$}}
    \label{thm:順列の総数②}

    \begin{align}
        _{\bm{n}}\mathbf{P}_{\bm{r}} \bm{=} \dfrac{\,\bm{n!}\,}{\,\bm{(n-r)}\bm{!}\,}
        \label{eq:順列の総数②}
    \end{align}
\end{theorembox}
\vspace{1em}

Theorem\ref{thm:順列の総数②}が $r = 0, r = n$ のときも成り立つように, ${}_{\bm{n}}\mathbf{P}_{\bm{0}} \bm{= 1}$ , $\bm{0! = 1}$ と定めることとする.

\newpage

\begin{definitionbox}[def:円順列]{\textbf{円順列}}
    \label{def:円順列}
    いくつかのものを円形に並べる順列を \textbf{円順列} という.

    円順列では, \uwave{回転して並びが同じになるものは同じ並び方とみなす}.
\end{definitionbox}

\begin{theorembox}[thm:円順列の総数]{\textbf{円順列の総数}}
    \label{thm:円順列の総数}
    異なる $n$ 個のものの円順列の総数について, 次のことがいえる.

    \begin{align}
        \dfrac{\,_{\bm{n}}\mathbf{P}_{\bm{n}}\,}{\,\bm{n}\,} \bm{= (n-1)!}
        \label{eq:円順列の総数}
    \end{align}
\end{theorembox}

% じゅず順列
\vspace{1em}

ここまでは, 異なるものだけを並べる順列を考えてきた. ここでは, 同じものを繰り返し使うことを許した場合の順列を考えてみよう.

% 重複順列
\begin{definitionbox}[def:重複順列]{\textbf{重複順列}}
    \label{def:重複順列}
    一般に, \uwave{異なる $n$ 種類のものから重複を許して $r$ 個取って} 並べる順列を
    \bm{$n$} \textbf{個から} \bm{$r$} \textbf{個取る重複順列} という.

    重複順列では, $r \leqq n$ とは限らず, $r > n$ であってもよい.
\end{definitionbox}

\begin{theorembox}[thm:重複順列の総数]{\textbf{重複順列の総数}}
    \label{thm:重複順列の総数}
    重複順列の総数について, 次のことがいえる.

    \begin{align}
        n \text{個から} r \text{個取る重複順列の総数は} \qquad \bm{n^r}
        \label{eq:重複順列の総数}
    \end{align}
\end{theorembox}
\newpage

\subsection{組合せ}

% 組合わせ
\begin{definitionbox}[def:組合せ]{\textbf{組合せ}}
    \label{def:組合せ}
    一般に, \uwave{異なる $n$ 個のものから異なる $r$ 個を取り出して} 作る組合せを
    \begin{center}
        $\bm{n}$\textbf{個から}$\bm{r}$\textbf{個取る組合せ}
    \end{center}
    といい, その総数を $_{\bm{n}}\mathbf{C}_{\bm{r}}$ $^{*}$ で表す. ただし, $r \leqq n$ とする.
\end{definitionbox}
\vspace{1em}

ここで, Permutation と Combination の違いを確認しておこう.

まずは, 日本語の意味の違いから. Permutation は「順列」, Combination は「組合せ」である.

次に, 数学的な視点から, Permutation と Combination の違いを考えてみよう.















\vspace{1em}

\begin{theorembox}[thm:組合せの総数]{\textbf{組合せの総数 $_{\bm{n}}\mathbf{C}_{\bm{r}}$}}
    $n$ 個から $r$ 個取る組合せの総数 $_{n}\mathrm{C}_{r}$ は次の式で表される.

    \begin{align}
        {}_{\bm{n}}\mathrm{C}_{\bm{r}} \bm{=} \dfrac{\,{}_{\bm{n}}\mathrm{P}_{\bm{r}}\,}{\,\bm{r!}\,} \bm{=}
        \dfrac{\,\bm{n(n-1)(n-2)\cdots\cdots(n-r+1)}\,}{\,\bm{r(r-1)(r-2)\cdots\cdots \cdot 3 \cdot 2 \cdot 1}\,}
        \label{eq:組合せの総数}
    \end{align}
\end{theorembox}
\vspace{1em}

Theorem\ref{eq:組合せの総数}において, 特に ${}_n\mathrm{C}_1 = n$ , ${}_n\mathrm{C}_n = 1$ である.

また, 式\ref{eq:順列の総数②}より, ${}_{\bm{n}}\mathbf{C}_{\bm{r}} \bm{=} \dfrac{\,\bm{n!}\,}{\,\bm{r!(n-r)!}\,}$
と表すこともできる. ただし, ${}_{\bm{n}}\mathbf{C}_{\bm{0}} \bm{=} 1$ と定めることとする.
\vspace{1em}

\begin{theorembox}[thm:]{\textbf{$_{\bm{n}}\mathbf{C}_{\bm{r}}$ の性質}}
    一般に, $n$ 個から $r$ 個取る組合せの総数は, $n$ 個から $(n-r)$ 個取る組合わせの総数に等しい.
    すなわち, 次の等式が成り立つ.
    \begin{align}
        {}_{\bm{n}}\mathbf{C}_{\bm{r}} \bm{=} {}_{\bm{n}}\mathbf{C}_{\bm{n-r}}
    \end{align}
\end{theorembox}

\begin{theorembox}[thm:同じものを含む順列の総数]{\textbf{同じものを含む順列の総数}}
    $a$ が $p$ 個, $b$ が $q$ 個, $c$ が $r$ 個あるとき, それら全部を1列に並べる順列の総数は

    \begin{align}
        {}_{n}\mathrm{C}_{p} \times {}_{n-p}\mathrm{C}_{q} = \dfrac{\,\bm{n!}\,}{\,\bm{p!\;q!\;r!}\,} \qquad \qquad
        \text{ただし} \qquad p + q + r = n
        \label{eq:同じものを含む順列の総数}
    \end{align}
\end{theorembox}
\vspace{1em}

$r = 0$ のときは, ${}_{n-p}\mathrm{C}_{q} = 1$ であり, 順列の総数は $\dfrac{\,n!\,}{\,p!\;q!\,}$ である.


\section{確率}

% 第2節 確率
\section{確率}
\subsection{事象と確率}

% 確率
    \begin{definitionbox}[def:確率]{\textbf{確率}}
        ある事柄が起こることが期待される程度を表す数値を \textbf{確率} という.
    \end{definitionbox}

    \begin{definitionbox}[def:試行と事象]{\textbf{試行と事象}}
        「さいころを投げる」とか「くじを引く」などのように, 同じ条件のもとで繰り返すことができ,
        その結果が偶然によって決まる実験や観測を \textbf{試行} という.

        また, 試行の結果として起こる事柄を \textbf{事象} という.
    \end{definitionbox}

    \begin{definitionbox}[def:全事象と根元事象]{\textbf{全事象と根元事象}}
        1つの試行において, 起こりうる結果全体を集合 $U$ で表すとき, その試行におけるどの事象も,
        $U$ の部分集合で表すことができる. $U$ 自身で表される事象を \textbf{全事象} ,
        $U$ のただ1つの要素からなる集合で表される事象を \textbf{根元事象} という.
    \end{definitionbox}

    \begin{definitionbox}[def:同様に確からしい]{\textbf{同様に確からしい}}
        1つの試行において, ある事象 $A$ の起こる確率を $P(A)$ で表す. $^{*}$

        ある試行において, どの根元事象が起こることも同程度に期待できるとき, これらの根元事象は \textbf{同様に確からしい} という.$^{**}$
    \end{definitionbox}

    \begin{theorembox}[thm:事象$A$の起こる確率]{\textbf{事象$A$の起こる確率}}
        同様に確からしい試行において, 起こりうるすべての場合の数を $N$ , 事象 $A$ の起こる場合の数を $a$ とするとき,
        事象 $A$ の起こる確率 $\bm{P(A)}$ は次の式で表される.

        \begin{align}
            \bm{P(A)} \bm{=} \dfrac{\text{\textbf{事象 $A$ の起こる場合の数}}}{\text{\textbf{起こりうるすべての場合の数}}} \bm{=} \dfrac{\,\bm{a}\,}{\,\bm{N}\,}
            \label{eq:事象$A$の起こる確率}
        \end{align}
    \end{theorembox}

    \vfill

    \noindent
    \rule{\textwidth}{0.4pt}

    \noindent
    $^{*}$ $P(A)$ の $P$ は,  確率を意味するProbability の頭文字である. また, 事象$A$の起こる確率を単に \textbf{事象 $A$ の確率} ということもある.

    \noindent
    $^{**}$ ここから取り上げる試行では, \uwave{全事象 $U$ におけるすべての根元事象は同様に確からしいものとする}.
\newpage

% 例題8
\begin{example}[【教p.47 例題8】]
    A, B の2人を含む6人のリレー選手がいる。走る順番をくじで決めるとき、A が1番目、B が6番目になる確率を求めよ。
\end{example}

\begin{proof}\mbox{}\\
    まず、起こりうるすべての場合の数を考える。

    6人全員の並び順は, $\answermath{6!}$ 通りある.

    次に、求めたい場合の数を考える。

    A が1番目、B が6番目のとき, A, B 以外の4人の並び順は, $\answermath{4!}$ 通りある.

    よって、求める確率は

    \begin{alignanswer*}
        \dfrac{\answermath{4!}}{\answermath{6!}} = \dfrac{1}{30}
    \end{alignanswer*}
\end{proof}

% 例題9
\begin{example}[【教p.47 例題9】]
    赤玉 $4$ 個, 白玉 $5$ 個の入った袋から、$2$ の玉を同時に取り出すとき、赤玉 $1$ 個、白玉 $1$ 個が出る確率を求めよ。
\end{example}

\begin{proof}\mbox{}\\
    まず、起こりうるすべての場合の数を考える。
    全部の $9$ 個から $2$ 個取る組合せは, $\answermath{{}_{9}\mathrm{C}_{2}}$ 通りある.

    次に、求めたい場合の数を考える。

    赤玉 $4$ 個から $1$ 個、白玉 $5$ 個から $1$ 個取る組合せは,
    $\answermath{{}_{4}\mathrm{C}_{1} \times {}_{5}\mathrm{C}_{1}}$ 通りある.

    よって、求める確率は

    \begin{alignanswer*}
        \dfrac{\answermath{{}_{4}\mathrm{C}_{1} \times {}_{5}\mathrm{C}_{1}}}{\answermath{{}_{9}\mathrm{C}_{2}}}
        = 4 \times 5 \times \dfrac{2 \cdot 1}{9 \cdot 8} = \dfrac{5}{9}
    \end{alignanswer*}
\end{proof}
\newpage

\subsection{確率の基本性質}

確率の基本性質を議論する前に、事象 $A$, $B$ に対して、次のような事象を定義しておく。

    \begin{definitionbox}[def:積事象と和事象]{\textbf{積事象と和事象}}
        事象 $A$ と $B$ がともに起こる事象を $A$ , $B$ の \textbf{積事象} といい, $\bm{A \cap B}$ で表す.

        一方, 事象 $A$ または $B$ が起こる事象を $A$ と $B$ の \textbf{和事象} といい, $\bm{A \cup B}$ で表す.
    \end{definitionbox}

    \begin{definitionbox}[def:排反事象]{\textbf{排反事象}}
        2つの事象 $A$ 、 $B$ が決して同時に起こらないとき、
        \[A, B \text{は互いに} \textbf{排反} \text{である}\]
        または
        \[A, B \text{は互いに} \textbf{排反事象} \text{である}\]
        という。
    \end{definitionbox}

    \begin{definitionbox}[def:空事象]{\textbf{空事象}}
        $A$ 、 $B$ が互いに排反であることは、$A \cap B = \emptyset$ であることと同値である。

        空集合 $\emptyset$ で表される事象を \textbf{空事象} という。空事象は決して起こらない事象である。
    \end{definitionbox}

    \begin{theorembox}[thm:確率の基本性質]{\textbf{確率の基本性質}}
        事象 $A$ の確率 $P(A)$ について, 次の性質が成り立つ.

        \begin{align}
            &0 \leq P(A) \leq 1
            \label{eq:確率の範囲}\\
            &P(U) = 1, \quad P(\emptyset) = 0
            \label{eq:全事象と空事象}\\
            &P(\bar{A}) = 1 - P(A)
            \label{eq:余事象の確率}
        \end{align}

        ここで, $U$ は全事象, $\emptyset$ は空事象, $\bar{A}$ は事象 $A$ の余事象を表す.
    \end{theorembox}

% \subsection{確率の加法定理}

    \begin{theorembox}[thm:確率の加法定理]{\textbf{確率の加法定理}}
        2つの事象 $A$, $B$ について, 次が成り立つ.

        \begin{align}
            P(A \cup B) = P(A) + P(B) - P(A \cap B)
            \label{eq:確率の加法定理}
        \end{align}

        特に, $A$ と $B$ が互いに排反である場合($A \cap B = \emptyset$), 次が成り立つ.

        \begin{align}
            P(A \cup B) = P(A) + P(B)
            \label{eq:排反事象の確率}
        \end{align}
    \end{theorembox}

\subsection{独立な試行と確率}

    \begin{theorembox}[thm:独立な試行の確率]{\textbf{独立な試行の確率}}
        2つの試行 $T_1$, $T_2$ が独立であるとき, 事象 $A$, $B$ について次が成り立つ.

        \begin{align}
            P(A \cap B) = P(A) \cdot P(B)
            \label{eq:独立な試行の確率}
        \end{align}
    \end{theorembox}

% \subsection{反復試行の確率}

    \begin{theorembox}[thm:反復試行の確率]{\textbf{反復試行の確率}}
        1回の試行で事象 $A$ が起こる確率を $p$ とする.
        この試行を $n$ 回繰り返すとき, 事象 $A$ がちょうど $r$ 回起こる確率は次の式で表される.

        \begin{align}
            {}_n\mathrm{C}_r \cdot p^r \cdot (1-p)^{n-r}
            \label{eq:反復試行の確率}
        \end{align}
    \end{theorembox}

\subsection{条件付き確率}

    \begin{definitionbox}{}{\textbf{条件付き確率}}
        事象 $A$ が起こったという条件のもとで事象 $B$ が起こる\textbf{条件付き確率} $P_A(B)$ は次のように定義される.

        \begin{align}
            P_A(B) = \frac{\,P(A \cap B)\,}{\,P(A)\,} \quad (P(A) > 0)
            \label{eq:条件付き確率}
        \end{align}
    \end{definitionbox}

    \begin{theorembox}[thm:確率の乗法定理]{\textbf{確率の乗法定理}}
        2つの事象 $A$, $B$ について, 次が成り立つ.

        \begin{align}
            P(A \cap B) = P(A) \cdot P_A(B)
            \label{eq:確率の乗法定理}
        \end{align}
    \end{theorembox}


    \subsection{期待値}

    \begin{definitionbox}{}{\textbf{条件付き確率}}
        事象 $A$ が起こったという条件のもとで事象 $B$ が起こる\textbf{条件付き確率} $P_A(B)$ は次のように定義される.

        \begin{align}
            P_A(B) = \frac{\,P(A \cap B)\,}{\,P(A)\,} \quad (P(A) > 0)
            % \label{eq:条件付き確率}
        \end{align}
    \end{definitionbox}

    \begin{theorembox}[thm:確率の乗法定理]{\textbf{確率の乗法定理}}
        2つの事象 $A$, $B$ について, 次が成り立つ.

        \begin{align}
            P(A \cap B) = P(A) \cdot P_A(B)
            % \label{eq:確率の乗法定理}
        \end{align}
    \end{theorembox}





\end{document}

